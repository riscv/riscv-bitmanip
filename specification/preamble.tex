% Package includes

\usepackage{graphicx}
\usepackage{geometry}
\usepackage{array}
\usepackage{colortbl}
\usepackage{hyperref}
\usepackage{placeins}
\usepackage{longtable}
\usepackage{multirow}
\usepackage{float}
\usepackage{xcolor}
\usepackage{listings}
\usepackage{comment}
\usepackage{enumitem}
\usepackage{verbatimbox}

\usepackage[olditem,oldenum]{paralist}

% Setup margins

\setlength{\topmargin}{-0.5in}
\setlength{\textheight}{9in}
\setlength{\oddsidemargin}{0in}
\setlength{\evensidemargin}{0in}
\setlength{\textwidth}{6.5in}

% Useful macros

\newcommand{\note}[1]{{\bf [ NOTE: #1 ]}}
\newcommand{\fixme}[1]{{\bf [ FIXME: #1 ]}}
\newcommand{\todo}[1]{\marginpar{\footnotesize #1}}

\newcommand{\wunits}[2]{\mbox{#1\,#2}}
\newcommand{\um}{\mbox{$\mu$m}}
\newcommand{\xum}[1]{\wunits{#1}{\um}}
\newcommand{\by}[2]{\mbox{#1$\times$#2}}
\newcommand{\byby}[3]{\mbox{#1$\times$#2$\times$#3}}

\newlength\savedwidth
\newcommand\whline[1]{%
  \noalign{%
    \global\savedwidth\arrayrulewidth\global\arrayrulewidth 1.5pt%
  }%
  \cline{#1}%
  \noalign{\vskip\arrayrulewidth}%
  \noalign{\global\arrayrulewidth\savedwidth}%
}

% Custom list environments

\newenvironment{tightlist}
{\begin{itemize}
 \setlength{\parsep}{0pt}
 \setlength{\itemsep}{-2pt}}
{\end{itemize}}

\newenvironment{titledtightlist}[1]
{\noindent
 ~~\textbf{#1}
 \begin{itemize}
 \setlength{\parsep}{0pt}
 \setlength{\itemsep}{-2pt}}
{\end{itemize}}

\newenvironment{commentary}
{ \vspace{-0.2in}
  \begin{quotation}
  \noindent
  \small \em
  \rule{\linewidth}{1pt}\\
}
{ 
  \end{quotation}
  \vspace{-0.2in}
}

\newenvironment{samepage-commentary}
{\begin{samepage} \begin{commentary}}
{\end{commentary} \end{samepage}}

\newenvironment{discussion}
{ \vspace{-0.2in}
  \begin{quotation}
  \noindent
  \small \em 
  \rule{\linewidth}{1pt} \\
  {\bf Discussion:}
}
{ 
  \end{quotation}
  \vspace{-0.2in}
}

% Other commands and parameters

\pagestyle{myheadings}
\setlength{\parindent}{0in}
\setlength{\parskip}{10pt}
\sloppy

% Commands for register format figures.

% New column types to use in tabular environment for instruction formats.
% Allocate 0.18in per bit.
\newcolumntype{I}{>{\centering\arraybackslash}p{0.18in}}
% Two-bit centered column.
\newcolumntype{W}{>{\centering\arraybackslash}p{0.36in}}
% Three-bit centered column.
\newcolumntype{F}{>{\centering\arraybackslash}p{0.54in}}
% Four-bit centered column.
\newcolumntype{Y}{>{\centering\arraybackslash}p{0.72in}}
% Five-bit centered column.
\newcolumntype{R}{>{\centering\arraybackslash}p{0.9in}}
% Six-bit centered column.
\newcolumntype{S}{>{\centering\arraybackslash}p{1.08in}}
% Seven-bit centered column.
\newcolumntype{O}{>{\centering\arraybackslash}p{1.26in}}
% Eight-bit centered column.
\newcolumntype{E}{>{\centering\arraybackslash}p{1.44in}}
% Ten-bit centered column.
\newcolumntype{T}{>{\centering\arraybackslash}p{1.8in}}
% Twelve-bit centered column.
\newcolumntype{M}{>{\centering\arraybackslash}p{2.2in}}
% Sixteen-bit centered column.
\newcolumntype{K}{>{\centering\arraybackslash}p{2.88in}}
% Twenty-bit centered column.
\newcolumntype{U}{>{\centering\arraybackslash}p{3.6in}}
% Twenty-bit centered column.
\newcolumntype{L}{>{\centering\arraybackslash}p{3.6in}}
% Twenty-five-bit centered column.
\newcolumntype{J}{>{\centering\arraybackslash}p{4.5in}}

\newcommand{\instbit}[1]{\mbox{\scriptsize #1}}
\newcommand{\instbitrange}[2]{~\instbit{#1} \hfill \instbit{#2}~}
\newcommand{\reglabel}[1]{\hfill {\tt #1}\hfill\ }

\newcommand{\wiri}{\textbf{WIRI}}
\newcommand{\wpri}{\textbf{WPRI}}
\newcommand{\wlrl}{\textbf{WLRL}}
\newcommand{\warl}{\textbf{WARL}}



