\chapter{RISC-V Bitmanip Extension}
\label{bext}

In the proposals provided in this section, the C code examples are for
illustration purposes. They are not optimal implementations, but are intended
to specify the desired functionality. See Chapter~\ref{fastc} for fast C code
for use in emulators.

The sections on encodings are mere placeholders.

%%%%%%%%%%%%%%%%%%%%%%%%%%%%%%%%%%%%%%%%%%%%%%%%%%%%%%%%%%%%%%%%%%%%%%%%%%%%%%%%%%%%%%%%%%%

\section{Count Leading/Trailing Zeros (\texttt{clz, ctz})}

The {\tt clz} operation counts the number of 0 bits at the MSB end of the
argument.  That is, the number of 0 bits before the first 1 bit counting from
the most significant bit. If the input is 0, the output is XLEN. If the input
is -1, the output is 0.

The {\tt ctz} operation counts the number of 0 bits at the LSB end of the
argument. If the input is 0, the output is XLEN. If the input is -1, the
output is 0.

\input{bextcref-clz-ctz}

\vspace{-0.3in}
\begin{center}
\begin{tabular}{M@{}R@{}S@{}R@{}O}
\\
\instbitrange{31}{20} &
\instbitrange{19}{15} &
\instbitrange{14}{12} &
\instbitrange{11}{7} &
\instbitrange{6}{0} \\
\hline
\multicolumn{1}{|c|}{imm[11:0]} &
\multicolumn{1}{c|}{rs1} &
\multicolumn{1}{c|}{funct3} &
\multicolumn{1}{c|}{rd} &
\multicolumn{1}{c|}{opcode} \\
\hline
12 & 5 & 3 & 5 & 7 \\
???????????? & src & CLZ  & dest & OP-IMM \\
???????????? & src & CTZ  & dest & OP-IMM \\
???????????? & src & CLZW  & dest & OP-IMM-32 \\
???????????? & src & CTZW  & dest & OP-IMM-32 \\
\end{tabular}
\end{center}


One possible encoding for \texttt{clz} and \texttt{ctz} is as standard I-type opcodes
somewhere in the brownfield surrounding the shift-immediate instructions.

% \subsection{References}
%
% https://en.wikipedia.org/wiki/Find\_first\_set\#CLZ
%
% https://fgiesen.wordpress.com/2013/10/18/bit-scanning-equivalencies/

%%%%%%%%%%%%%%%%%%%%%%%%%%%%%%%%%%%%%%%%%%%%%%%%%%%%%%%%%%%%%%%%%%%%%%%%%%%%%%%%%%%%%%%%%%%

\section{Count Bits Set (\texttt{pcnt})}

This instruction counts the number of 1 bits in a register. This operations is known as
population count, popcount, sideways sum, bit summation, or Hamming weight.~\cite{HammingWeight,Warren12}

\input{bextcref-pcnt}

\vspace{-0.3in}
\begin{center}
\begin{tabular}{M@{}R@{}S@{}R@{}O}
\\
\instbitrange{31}{20} &
\instbitrange{19}{15} &
\instbitrange{14}{12} &
\instbitrange{11}{7} &
\instbitrange{6}{0} \\
\hline
\multicolumn{1}{|c|}{imm[11:0]} &
\multicolumn{1}{c|}{rs1} &
\multicolumn{1}{c|}{funct3} &
\multicolumn{1}{c|}{rd} &
\multicolumn{1}{c|}{opcode} \\
\hline
12 & 5 & 3 & 5 & 7 \\
???????????? & src & PCNT  & dest & OP-IMM \\
???????????? & src & PCNTW  & dest & OP-IMM-32 \\
\end{tabular}
\end{center}


One possible encoding for \texttt{pcnt} is as a standard I-type opcode somewhere
in the brownfield surrounding the shift-immediate instructions.

%%%%%%%%%%%%%%%%%%%%%%%%%%%%%%%%%%%%%%%%%%%%%%%%%%%%%%%%%%%%%%%%%%%%%%%%%%%%%%%%%%%%%%%%%%%

\section{And-with-complement (\texttt{andc})}

This instruction implements the and-with-complement operation.

\input{bextcref-andc}

Other with-complement operations ({\tt orc, nand, nor}, etc) can be implemented
by combining {\tt not} ({\tt c.not}) with the base ALU operation. (Which can
fit in 32 bit when using two compressed instructions.) Only and-with-complement
occurs frequently enough to warrant a dedicated instruction.

\vspace{-0.3in}
\begin{center}
\begin{tabular}{S@{}R@{}R@{}S@{}R@{}O}
\\
\instbitrange{31}{25} &
\instbitrange{24}{20} &
\instbitrange{19}{15} &
\instbitrange{14}{12} &
\instbitrange{11}{7} &
\instbitrange{6}{0} \\
\hline
\multicolumn{1}{|c|}{funct7} &
\multicolumn{1}{c|}{rs2} &
\multicolumn{1}{c|}{rs1} &
\multicolumn{1}{c|}{funct3} &
\multicolumn{1}{c|}{rd} &
\multicolumn{1}{c|}{opcode} \\
\hline
7 & 5 & 5 & 3 & 5 & 7 \\
??????? & src2 & src1 & ANDC  & dest & OP    \\
% ??????? & src2 & src1 & ANDCW & dest & OP-32 \\
\end{tabular}
\end{center}


% \subsection{Justification}
%
% http://svn.clifford.at/handicraft/2017/bitcode/
%
% \subsection{References}
%
% http://www.hackersdelight.org/basics2.pdf

%%%%%%%%%%%%%%%%%%%%%%%%%%%%%%%%%%%%%%%%%%%%%%%%%%%%%%%%%%%%%%%%%%%%%%%%%%%%%%%%%%%%%%%%%%%

\section{Shift Ones (Left/Right) (\texttt{slo,\ sloi,\ sro,\ sroi})}

These instructions are similar to shift-logical operations from the base
spec, except instead of shifting in zeros, they shift in ones.

\input{bextcref-sxo}

\vspace{-0.3in}
\begin{center}
\begin{tabular}{S@{}R@{}R@{}S@{}R@{}O}
\\
\instbitrange{31}{25} &
\instbitrange{24}{20} &
\instbitrange{19}{15} &
\instbitrange{14}{12} &
\instbitrange{11}{7} &
\instbitrange{6}{0} \\
\hline
\multicolumn{1}{|c|}{funct7} &
\multicolumn{1}{c|}{rs2} &
\multicolumn{1}{c|}{rs1} &
\multicolumn{1}{c|}{funct3} &
\multicolumn{1}{c|}{rd} &
\multicolumn{1}{c|}{opcode} \\
\hline
7 & 5 & 5 & 3 & 5 & 7 \\
10????? & src2 & src1 & SRO    & dest & OP    \\
10????? & src2 & src1 & SLO    & dest & OP    \\
10????? & src2 & src1 & SROW   & dest & OP-32 \\
10????? & src2 & src1 & SLOW   & dest & OP-32 \\
\end{tabular}
\end{center}

\vspace{-0.4in}
\begin{center}
\begin{tabular}{S@{}R@{}R@{}S@{}R@{}O}
\\
\instbitrange{31}{27} &
\instbitrange{26}{20} &
\instbitrange{19}{15} &
\instbitrange{14}{12} &
\instbitrange{11}{7} &
\instbitrange{6}{0} \\
\hline
\multicolumn{1}{|c|}{imm[11:7]} &
\multicolumn{1}{c|}{imm[6:0]} &
\multicolumn{1}{c|}{rs1} &
\multicolumn{1}{c|}{funct3} &
\multicolumn{1}{c|}{rd} &
\multicolumn{1}{c|}{opcode} \\
\hline
5 & 7 & 5 & 3 & 5 & 7 \\
10??? & shamt & src & SLOI & dest & OP-IMM \\
10??? & shamt & src & SROI & dest & OP-IMM \\
\end{tabular}
\end{center}

\vspace{-0.4in}
\begin{center}
\begin{tabular}{S@{}R@{}R@{}S@{}R@{}O}
\\
\instbitrange{31}{25} &
\instbitrange{24}{20} &
\instbitrange{19}{15} &
\instbitrange{14}{12} &
\instbitrange{11}{7} &
\instbitrange{6}{0} \\
\hline
\multicolumn{1}{|c|}{imm[11:5]} &
\multicolumn{1}{c|}{imm[4:0]} &
\multicolumn{1}{c|}{rs1} &
\multicolumn{1}{c|}{funct3} &
\multicolumn{1}{c|}{rd} &
\multicolumn{1}{c|}{opcode} \\
\hline
7 & 5 & 5 & 3 & 5 & 7 \\
10????? & shamt & src & SLOIW & dest & OP-IMM-32 \\
10????? & shamt & src & SROIW & dest & OP-IMM-32 \\
\end{tabular}
\end{center}


\texttt{s(l/r)o(i)} is encoded similarly to the logical shifts in the
base spec. However, the spec of the entire family of instructions is
changed so that the high bit of the instruction indicates the value to
be inserted during a shift. This means that a \texttt{sloi} instruction
can be encoded similarly to an \texttt{slli} instruction, but with a 1
in the highest bit of the encoded instruction. This encoding is
backwards compatible with the definition for the shifts in the base
spec, but allows for simple addition of a ones-insert.

When implementing this circuit, the only change in the ALU over a
standard logical shift is that the value shifted in is not zero, but is
a 1-bit register value that has been forwarded from the high bit of the
instruction decode. This creates the desired behavior on both logical
zero-shifts and logical ones-shifts.

%%%%%%%%%%%%%%%%%%%%%%%%%%%%%%%%%%%%%%%%%%%%%%%%%%%%%%%%%%%%%%%%%%%%%%%%%%%%%%%%%%%%%%%%%%%

\section{Rotate (Left/Right) (\texttt{rol,\ ror,\ rori})}

These instructions are similar to shift-logical operations from the base
spec, except they shift in the values from the opposite side of the
register, in order. This is also called `circular shift'.

\input{bextcref-rox}

\vspace{-0.3in}
\begin{center}
\begin{tabular}{S@{}R@{}R@{}S@{}R@{}O}
\\
\instbitrange{31}{25} &
\instbitrange{24}{20} &
\instbitrange{19}{15} &
\instbitrange{14}{12} &
\instbitrange{11}{7} &
\instbitrange{6}{0} \\
\hline
\multicolumn{1}{|c|}{funct7} &
\multicolumn{1}{c|}{rs2} &
\multicolumn{1}{c|}{rs1} &
\multicolumn{1}{c|}{funct3} &
\multicolumn{1}{c|}{rd} &
\multicolumn{1}{c|}{opcode} \\
\hline
7 & 5 & 5 & 3 & 5 & 7 \\
11????? & src2 & src1 & ROR    & dest & OP    \\
11????? & src2 & src1 & ROL    & dest & OP    \\
11????? & src2 & src1 & RORW   & dest & OP-32 \\
11????? & src2 & src1 & ROLW   & dest & OP-32 \\
\end{tabular}
\end{center}

\vspace{-0.4in}
\begin{center}
\begin{tabular}{S@{}R@{}R@{}S@{}R@{}O}
\\
\instbitrange{31}{27} &
\instbitrange{26}{20} &
\instbitrange{19}{15} &
\instbitrange{14}{12} &
\instbitrange{11}{7} &
\instbitrange{6}{0} \\
\hline
\multicolumn{1}{|c|}{imm[11:7]} &
\multicolumn{1}{c|}{imm[6:0]} &
\multicolumn{1}{c|}{rs1} &
\multicolumn{1}{c|}{funct3} &
\multicolumn{1}{c|}{rd} &
\multicolumn{1}{c|}{opcode} \\
\hline
5 & 7 & 5 & 3 & 5 & 7 \\
11??? & shamt & src & RORI & dest & OP-IMM \\
\end{tabular}
\end{center}

\vspace{-0.4in}
\begin{center}
\begin{tabular}{S@{}R@{}R@{}S@{}R@{}O}
\\
\instbitrange{31}{25} &
\instbitrange{24}{20} &
\instbitrange{19}{15} &
\instbitrange{14}{12} &
\instbitrange{11}{7} &
\instbitrange{6}{0} \\
\hline
\multicolumn{1}{|c|}{imm[11:5]} &
\multicolumn{1}{c|}{imm[4:0]} &
\multicolumn{1}{c|}{rs1} &
\multicolumn{1}{c|}{funct3} &
\multicolumn{1}{c|}{rd} &
\multicolumn{1}{c|}{opcode} \\
\hline
7 & 5 & 5 & 3 & 5 & 7 \\
11????? & shamt & src & RORIW & dest & OP-IMM-32 \\
\end{tabular}
\end{center}


Rotate shift is implemented very similarly to the other shift
instructions. One possible way to encode it is to re-use the way that
bit 30 in the instruction encoding selects `arithmetic shift' when bit
31 is zero (signalling a logical-zero shift). We can re-use this so that
when bit 31 is set (signalling a logical-ones shift), if bit 30 is also
set, then we are doing a rotate. The following table summarizes the
behavior. The generalized reverse instructions can be encoded using the
bit pattern that would otherwise encode an ``Arithmetic Left Shift''
(which is an operation that does not exist). Likewise, the generalized zip
instruction can be encoded using the bit pattern that would otherwise
encode an ``Rotate left immediate''.

\begin{center}
\begin{tabular}{lll}
Bit 31 & Bit 30 & Meaning \\
\hline
0 & 0 & Logical Shift-Zeros \\
0 & 1 & Arithmetic Shift \\
1 & 0 & Logical Shift-Ones \\
1 & 1 & Rotate \\
\end{tabular}
\end{center}

%%%%%%%%%%%%%%%%%%%%%%%%%%%%%%%%%%%%%%%%%%%%%%%%%%%%%%%%%%%%%%%%%%%%%%%%%%%%%%%%%%%%%%%%%%%

\begin{figure}[t]
\begin{center}
\input{bextcref-printperm-ror}
\end{center}
\caption{\texttt{ror} permutation network}
\label{permnet-ror}
\end{figure}

\section{Generalized Reverse (\texttt{grev,\ grevi})}
\label{grev}

This instruction provides a single hardware instruction that can implement all
of byte-order swap, bitwise reversal, short-order-swap, word-order-swap
(RV64), nibble-order swap, bitwise reversal in a byte, etc, all from a single
hardware instruction. It takes in a single register value and an immediate that
controls which function occurs, through controlling the levels in the recursive
tree at which reversals occur.

This operation iteratively checks each bit $i$ in rs2 from $i=0$ to
$\textrm{XLEN}-1$, and if the corresponding bit is set, swaps each adjacent
pair of $2^i$ bits.

\begin{figure}[t]
\begin{center}
\input{bextcref-printperm-grev}
\end{center}
\caption{\texttt{grev} permutation network}
\label{permnet-grev}
\end{figure}

\input{bextcref-grev}

The above pattern should be intuitive to understand in order to extend
this definition in an obvious manner for RV128.

\begin{table}[h]
\begin{small}
\begin{center}
\begin{tabular}{r l p{0.5in} r l p{0.3in} r l}

\multicolumn{2}{c}{RV32} & &
\multicolumn{5}{c}{RV64} \\

\cline{1-2}
\cline{4-8}

\multicolumn{1}{c}{shamt} & Instruction & &
\multicolumn{1}{c}{shamt} & Instruction & &
\multicolumn{1}{c}{shamt} & Instruction \\

\cline{1-2}
\cline{4-5}
\cline{7-8}

 0: 00000 & ---           &   &  0: 000000 & ---           &   & 32: 100000 & {\tt wswap} \\
 1: 00001 & {\tt brev.p}  &   &  1: 000001 & {\tt brev.p}  &   & 33: 100001 & ---         \\
 2: 00010 & {\tt pswap.n} &   &  2: 000010 & {\tt pswap.n} &   & 34: 100010 & ---         \\
 3: 00011 & {\tt brev.n}  &   &  3: 000011 & {\tt brev.n}  &   & 35: 100011 & ---         \\
 4: 00100 & {\tt nswap.b} &   &  4: 000100 & {\tt nswap.b} &   & 36: 100100 & ---         \\
 5: 00101 & ---           &   &  5: 000101 & ---           &   & 37: 100101 & ---         \\
 6: 00110 & {\tt pswap.b} &   &  6: 000110 & {\tt pswap.b} &   & 38: 100110 & ---         \\
 7: 00111 & {\tt brev.b}  &   &  7: 000111 & {\tt brev.b}  &   & 39: 100111 & ---         \\

\cline{1-2}
\cline{4-5}
\cline{7-8}

 8: 01000 & {\tt bswap.h} &   &  8: 001000 & {\tt bswap.h} &   & 40: 101000 & ---         \\
 9: 01001 & ---           &   &  9: 001001 & ---           &   & 41: 101001 & ---         \\
10: 01010 & ---           &   & 10: 001010 & ---           &   & 42: 101010 & ---         \\
11: 01011 & ---           &   & 11: 001011 & ---           &   & 43: 101011 & ---         \\
12: 01100 & {\tt nswap.h} &   & 12: 001100 & {\tt nswap.h} &   & 44: 101100 & ---         \\
13: 01101 & ---           &   & 13: 001101 & ---           &   & 45: 101101 & ---         \\
14: 01110 & {\tt pswap.h} &   & 14: 001110 & {\tt pswap.h} &   & 46: 101110 & ---         \\
15: 01111 & {\tt brev.h}  &   & 15: 001111 & {\tt brev.h}  &   & 47: 101111 & ---         \\

\cline{1-2}
\cline{4-5}
\cline{7-8}

16: 10000 & {\tt hswap}   &   & 16: 010000 & {\tt hswap.w} &   & 48: 110000 & {\tt hswap} \\
17: 10001 & ---           &   & 17: 010001 & ---           &   & 49: 110001 & ---         \\
18: 10010 & ---           &   & 18: 010010 & ---           &   & 50: 110010 & ---         \\
19: 10011 & ---           &   & 19: 010011 & ---           &   & 51: 110011 & ---         \\
20: 10100 & ---           &   & 20: 010100 & ---           &   & 52: 110100 & ---         \\
21: 10101 & ---           &   & 21: 010101 & ---           &   & 53: 110101 & ---         \\
22: 10110 & ---           &   & 22: 010110 & ---           &   & 54: 110110 & ---         \\
23: 10111 & ---           &   & 23: 010111 & ---           &   & 55: 110111 & ---         \\

\cline{1-2}
\cline{4-5}
\cline{7-8}

24: 11000 & {\tt bswap}   &   & 24: 011000 & {\tt bswap.w} &   & 56: 111000 & {\tt bswap} \\
25: 11001 & ---           &   & 25: 011001 & ---           &   & 57: 111001 & ---         \\
26: 11010 & ---           &   & 26: 011010 & ---           &   & 58: 111010 & ---         \\
27: 11011 & ---           &   & 27: 011011 & ---           &   & 59: 111011 & ---         \\
28: 11100 & {\tt nswap}   &   & 28: 011100 & {\tt nswap.w} &   & 60: 111100 & {\tt nswap} \\
29: 11101 & ---           &   & 29: 011101 & ---           &   & 61: 111101 & ---         \\
30: 11110 & {\tt pswap}   &   & 30: 011110 & {\tt pswap.w} &   & 62: 111110 & {\tt pswap} \\
31: 11111 & {\tt brev}    &   & 31: 011111 & {\tt brev.w}  &   & 63: 111111 & {\tt brev}  \\
\end{tabular}
\end{center}
\end{small}
\caption{Pseudo-instructions for {\tt grevi} instruction}
\label{grevi-modes}
\end{table}

The {\tt grev} operation can easily be implemented using a permutation
network with $log_2(\textrm{XLEN})$ stages. Figure~\ref{permnet-ror}
shows the permutation network for {\tt ror} for reference.
Figure~\ref{permnet-grev} shows the permutation network for {\tt grev}.

\vspace{-0.3in}
\begin{center}
\begin{tabular}{S@{}R@{}R@{}S@{}R@{}O}
\\
\instbitrange{31}{25} &
\instbitrange{24}{20} &
\instbitrange{19}{15} &
\instbitrange{14}{12} &
\instbitrange{11}{7} &
\instbitrange{6}{0} \\
\hline
\multicolumn{1}{|c|}{funct7} &
\multicolumn{1}{c|}{rs2} &
\multicolumn{1}{c|}{rs1} &
\multicolumn{1}{c|}{funct3} &
\multicolumn{1}{c|}{rd} &
\multicolumn{1}{c|}{opcode} \\
\hline
7 & 5 & 5 & 3 & 5 & 7 \\
??????? & src2 & src1 & GREV   & dest & OP    \\
% ??????? & src2 & src1 & GREVW  & dest & OP-32 \\
\end{tabular}
\end{center}

\vspace{-0.4in}
\begin{center}
\begin{tabular}{R@{}R@{}R@{}S@{}R@{}O}
\\
\instbitrange{31}{27} &
\instbitrange{26}{20} &
\instbitrange{19}{15} &
\instbitrange{14}{12} &
\instbitrange{11}{7} &
\instbitrange{6}{0} \\
\hline
\multicolumn{1}{|c|}{imm[11:7]} &
\multicolumn{1}{c|}{imm[6:0]} &
\multicolumn{1}{c|}{rs1} &
\multicolumn{1}{c|}{funct3} &
\multicolumn{1}{c|}{rd} &
\multicolumn{1}{c|}{opcode} \\
\hline
5 & 7 & 5 & 3 & 5 & 7 \\
????? & mode & src & GREVI  & dest & OP-IMM \\
\end{tabular}
\end{center}

% \vspace{-0.4in}
% \begin{center}
% \begin{tabular}{R@{}R@{}R@{}S@{}R@{}O}
% \\
% \instbitrange{31}{25} &
% \instbitrange{24}{20} &
% \instbitrange{19}{15} &
% \instbitrange{14}{12} &
% \instbitrange{11}{7} &
% \instbitrange{6}{0} \\
% \hline
% \multicolumn{1}{|c|}{imm[11:5]} &
% \multicolumn{1}{c|}{imm[4:0]} &
% \multicolumn{1}{c|}{rs1} &
% \multicolumn{1}{c|}{funct3} &
% \multicolumn{1}{c|}{rd} &
% \multicolumn{1}{c|}{opcode} \\
% \hline
% 7 & 5 & 5 & 3 & 5 & 7 \\
% ??????? & mode & src & GREVIW  & dest & OP-IMM-32 \\
% \end{tabular}
% \end{center}


\texttt{grev} is encoded as standard R-type opcode and \texttt{grevi} is
encoded as standard I-type opcode. \texttt{grev} and \texttt{grevi} can
use the instruction encoding for ``arithmetic shift left''.

% \subsection{References}
%
% Hackers Delight, Chapter 7.1, ``Generalized Bit Reversal'' in
%
% https://books.google.com/books?id=iBNKMspIlqEC\&lpg=PP1\&pg=RA1-SL20-PA2\#v=onepage\&q\&f=false
%
% http://hackersdelight.org/

%%%%%%%%%%%%%%%%%%%%%%%%%%%%%%%%%%%%%%%%%%%%%%%%%%%%%%%%%%%%%%%%%%%%%%%%%%%%%%%%%%%%%%%%%%%

\section{Generalized Shuffle (\texttt{shfl}, \texttt{unshfl}, \texttt{shfli}, \texttt{unshfli})}
\label{gzip}

Shuffle is the third bit permutation instruction in the RISC-V Bitmanip
extension, after rotary shift and generalized reverse. It implements a
generalization of the operation commonly known as perfect outer shuffle and its
inverse (shuffle/unshuffle), also known as zip/unzip or interlace/uninterlace.

Bit permutations can be understood as reversible functions on bit indices (i.e.
5 bit functions on RV32 and 6 bit functions on RV64).

\begin{center}
\begin{tabular}{l l}
Operation & Corresponding function on bit indices \\
\hline
Rotate shift & Addition modulo {\rm XLEN} \\
Generalized reverse & XOR with bitmask \\
Generalized shuffle & Bitpermutation \\
\end{tabular}
\end{center}

A generalized (un)shuffle operation has $log_2(\textrm{XLEN})-1$ control bits,
one for each pair of neighbouring bits in a bit index. When the bit is set,
generalized shuffle will swap the two index bits. The {\tt shfl} operation
performs this swaps in MSB-to-LSB order (performing a rotate left shift on
continuous regions of set control bits), and the {\tt unshfl} operation performs
the swaps in LSB-to-MSB order (performing a rotate right shift on continuous
regions of set control bits). Combining up to $log_2(\textrm{XLEN})$ of those
{\tt shfl}/{\tt unshfl} operations can implement any bitpermutation on the
bit indices.

The most common type of shuffle/unshuffle operation is one on an immediate
control value that only contains one continuous region of set bits. We call
those operations zip/unzip and provide pseudo-instructions for them.

Shuffle/unshuffle operations that only have individual bits set (not a continuous
region of two or more bits) are their own inverse. This means that a few of the
{\tt unshfli} opcodes are redundant with {\tt shfli} opcodes and therefore can
be reserved for encoding unary functions such as {\tt ctz}, {\tt clz}, and {\tt pcnt}.

\begin{table}[h]
\begin{small}
\begin{center}
\begin{tabular}{r c l l}
\multicolumn{1}{c}{shamt} &
\multicolumn{1}{c}{inv} &
Bit index rotations &
Pseudo-Instruction \\

\hline

 0: 0000 & 0 & no-op                            & ---                    \\
    0000 & 1 & no-op                            & {\it reserved}         \\
 1: 0001 & 0 & {\tt i[1] -> i[0]}               & {\tt zip.n, unzip.n}   \\
    0001 & 1 & {\it equivalent to 0001 0}       & {\it reserved}         \\
 2: 0010 & 0 & {\tt i[2] -> i[1]}               & {\tt zip2.b, unzip2.b} \\
    0010 & 1 & {\it equivalent to 0010 0}       & {\it reserved}         \\
 3: 0011 & 0 & {\tt i[2] -> i[0]}               & {\tt zip.b}            \\
    0011 & 1 & {\tt i[2] <- i[0]}               & {\tt unzip.b}          \\

\hline

 4: 0100 & 0 & {\tt i[3] -> i[2]}               & {\tt zip4.h, unzip4.h} \\
    0100 & 1 & {\it equivalent to 0100 0}       & {\it reserved}         \\
 5: 0101 & 0 & {\tt i[3] -> i[2], i[1] -> i[0]} & ---                    \\
    0101 & 1 & {\it equivalent to 0101 0}       & {\it reserved}         \\
 6: 0110 & 0 & {\tt i[3] -> i[1]}               & {\tt zip2.h}           \\
    0110 & 1 & {\tt i[3] <- i[1]}               & {\tt unzip2.h}         \\
 7: 0111 & 0 & {\tt i[3] -> i[0]}               & {\tt zip.h}            \\
    0111 & 1 & {\tt i[3] <- i[0]}               & {\tt unzip.h}          \\

\hline

 8: 1000 & 0 & {\tt i[4] -> i[3]}               & {\tt zip8, unzip8}     \\
    1000 & 1 & {\it equivalent to 1000 0}       & {\it reserved}         \\
 9: 1001 & 0 & {\tt i[4] -> i[3], i[1] -> i[0]} & ---                    \\
    1001 & 1 & {\it equivalent to 1001 0}       & {\it reserved}         \\
10: 1010 & 0 & {\tt i[4] -> i[3], i[2] -> i[1]} & ---                    \\
    1010 & 1 & {\it equivalent to 1010 0}       & {\it reserved}         \\
11: 1011 & 0 & {\tt i[4] -> i[3], i[2] -> i[0]} & ---                    \\
    1011 & 1 & {\tt i[4] <- i[3], i[2] <- i[0]} & ---                    \\

\hline

12: 1100 & 0 & {\tt i[4] -> i[2]}               & {\tt zip4}             \\
    1100 & 1 & {\tt i[4] <- i[2]}               & {\tt unzip4}           \\
13: 1101 & 0 & {\tt i[4] -> i[2], i[1] -> i[0]} & ---                    \\
    1101 & 1 & {\tt i[4] <- i[2], i[1] <- i[0]} & ---                    \\
14: 1110 & 0 & {\tt i[4] -> i[1]}               & {\tt zip2}             \\
    1110 & 1 & {\tt i[4] <- i[1]}               & {\tt unzip2}           \\
15: 1111 & 0 & {\tt i[4] -> i[0]}               & {\tt zip}              \\
    1111 & 1 & {\tt i[4] <- i[0]}               & {\tt unzip}            \\

\end{tabular}
\end{center}
\end{small}
\caption{RV32 modes and pseudo-instructions for {\tt shfli}/{\tt unshfli} instruction}
\label{gzip32-modes}
\end{table}

\begin{table}[h]
\begin{small}
\begin{center}
\begin{tabular}{r c l p{1in} r c l}
\multicolumn{1}{c}{shamt} &
\multicolumn{1}{c}{inv} &
Pseudo-Instruction & &
\multicolumn{1}{c}{shamt} &
\multicolumn{1}{c}{inv} &
Pseudo-Instruction \\

\cline{1-3}
\cline{5-7}

 0: 00000 & 0 & ---                      &   &   16: 10000 & 0 & {\tt zip16, unzip16}    \\
    00000 & 1 & {\it reserved}           &   &       10000 & 1 & {\it reserved}          \\
 1: 00001 & 0 & {\tt zip.n, unzip.n}     &   &   17: 10001 & 0 & ---                     \\
    00001 & 1 & {\it reserved}           &   &       10001 & 1 & {\it reserved}          \\
 2: 00010 & 0 & {\tt zip2.b, unzip2.b}   &   &   18: 10010 & 0 & ---                     \\
    00010 & 1 & {\it reserved}           &   &       10010 & 1 & {\it reserved}          \\
 3: 00011 & 0 & {\tt zip.b}              &   &   19: 10011 & 0 & ---                     \\
    00011 & 1 & {\tt unzip.b}            &   &       10011 & 1 & ---                     \\

\cline{1-3}
\cline{5-7}

 4: 00100 & 0 & {\tt zip4.h, unzip4.h}   &   &   20: 10100 & 0 & ---                     \\
    00100 & 1 & {\it reserved}           &   &       10100 & 1 & {\it reserved}          \\
 5: 00101 & 0 & ---                      &   &   21: 10101 & 0 & ---                     \\
    00101 & 1 & {\it reserved}           &   &       10101 & 1 & {\it reserved}          \\
 6: 00110 & 0 & {\tt zip2.h}             &   &   22: 10110 & 0 & ---                     \\
    00110 & 1 & {\tt unzip2.h}           &   &       10110 & 1 & ---                     \\
 7: 00111 & 0 & {\tt zip.h}              &   &   23: 10111 & 0 & ---                     \\
    00111 & 1 & {\tt unzip.h}            &   &       10111 & 1 & ---                     \\

\cline{1-3}
\cline{5-7}

 8: 01000 & 0 & {\tt zip8.w, unzip8.w}   &   &   24: 11000 & 0 & {\tt zip8}              \\
    01000 & 1 & {\it reserved}           &   &       11000 & 1 & {\tt unzip8}            \\
 9: 01001 & 0 & ---                      &   &   25: 11001 & 0 & ---                     \\
    01001 & 1 & {\it reserved}           &   &       11001 & 1 & ---                     \\
10: 01010 & 0 & ---                      &   &   26: 11010 & 0 & ---                     \\
    01010 & 1 & {\it reserved}           &   &       11010 & 1 & ---                     \\
11: 01011 & 0 & ---                      &   &   27: 11011 & 0 & ---                     \\
    01011 & 1 & ---                      &   &       11011 & 1 & ---                     \\

\cline{1-3}
\cline{5-7}

12: 01100 & 0 & {\tt zip4.w}             &   &   28: 11100 & 0 & {\tt zip4}              \\
    01100 & 1 & {\tt unzip4.w}           &   &       11100 & 1 & {\tt unzip4}            \\
13: 01101 & 0 & ---                      &   &   29: 11101 & 0 & ---                     \\
    01101 & 1 & ---                      &   &       11101 & 1 & ---                     \\
14: 01110 & 0 & {\tt zip2.w}             &   &   30: 11110 & 0 & {\tt zip2}              \\
    01110 & 1 & {\tt unzip2.w}           &   &       11110 & 1 & {\tt unzip2}            \\
15: 01111 & 0 & {\tt zip.w}              &   &   31: 11111 & 0 & {\tt zip}               \\
    01111 & 1 & {\tt unzip.w}            &   &       11111 & 1 & {\tt unzip}             \\

\end{tabular}
\end{center}
\end{small}
\caption{RV64 modes and pseudo-instructions for {\tt shfli}/{\tt unshfli} instruction}
\label{gzip64-modes}
\end{table}

\begin{figure}[t]
\begin{center}
\input{bextcref-printperm-gzip-noflip}
\end{center}
\caption{(un)shuffle permutation network without ``flip'' stages}
\label{permnet-gzip-noflip}
\end{figure}

Like GREV and rotate shift, the (un)shuffle instruction can be implemented using a short
sequence of elementary permutations, that are enabled or disabled by the shamt
bits. But (un)shuffle has one stage fewer than GREV. Thus shfli+unshfli together require
the same amount of encoding space as grevi.

\input{bextcref-gzip32}

Or for RV64:

\input{bextcref-gzip64}

The above pattern should be intuitive to understand in order to extend
this definition in an obvious manner for RV128.

Alternatively (un)shuffle) can be implemented in a single network with one more
stage than GREV, with the additional first and last stage executing a
permutation that effectively reverses the order of the inner stages. However,
since the inner stages only mux half of the bits in the word each, a hardware
implementation using this additional ``flip'' stages might actually be more
expensive than simply creating two networks.

\input{bextcref-gzip32-alt}

Figure~\ref{permnet-gzip-flip} shows the (un)shuffle permutation network with
``flip'' stages and Figure~\ref{permnet-gzip-noflip} shows the (un)shuffle
permutation network without ``flip'' stages.

\begin{figure}[t]
\begin{center}
\input{bextcref-printperm-gzip-flip}
\end{center}
\caption{(un)shuffle permutation network with ``flip'' stages}
\label{permnet-gzip-flip}
\end{figure}

The \texttt{zip} instruction with the upper half of its input cleared performs
the commonly needed ``fan-out'' operation. (Equivalent to {\tt bdep} with a
0x55555555 mask.) The \texttt{zip} instruction applied twice fans out the bits
in the lower quarter of the input word by a spacing of 4 bits.

For example, the following code calculates the bitwise prefix sum of the bits
in the lower byte of a 32 bit word on RV32:

\begin{verbatim}
  andi a0, a0, 0xff
  zip a0, a0
  zip a0, a0
  slli a1, a0, 4
  c.add a0, a1
  slli a1, a0, 8
  c.add a0, a1
  slli a1, a0, 16
  c.add a0, a1
\end{verbatim}

The final prefix sum is stored in the 8 nibbles of the {\tt a0} output word.

Similarly, the following code stores the indices of the set bits in the LSB
nibbles of the output word (with the LSB bit having index 1), with the unused
MSB nibbles in the output set to zero:

\begin{verbatim}
  andi a0, a0, 0xff
  zip a0, a0
  zip a0, a0
  slli a1, a0, 1
  or a0, a0, a1
  slli a1, a0, 2
  or a0, a0, a1
  li a1, 0x87654321
  and a1, a0, a1
  bext a0, a1, a0
\end{verbatim}

Other {\tt zip} modes can be used to ``fan-out'' in blocks of 2, 4, 8, or 16 bit.
{\tt zip} can be combined with {\tt grevi} to perform inner shuffles. For example
on RV64:

\begin{verbatim}
  li a0, 0x0000000012345678
  zip4 t0, a0    ; <- 0x0102030405060708
  nswap.b t1, t0 ; <- 0x1020304050607080
  zip8 t2, a0    ; <- 0x0012003400560078
  bswap.h t3, t2 ; <- 0x1200340056007800
  zip16 t4, a0   ; <- 0x0000123400005678
  hswap.w t5, t4 ; <- 0x1234000056780000
\end{verbatim}

Another application for the zip instruction is generating Morton
code~\cite[MortonCode].

\vspace{-0.3in}
\begin{center}
\begin{tabular}{S@{}R@{}R@{}S@{}R@{}O}
\\
\instbitrange{31}{25} &
\instbitrange{24}{20} &
\instbitrange{19}{15} &
\instbitrange{14}{12} &
\instbitrange{11}{7} &
\instbitrange{6}{0} \\
\hline
\multicolumn{1}{|c|}{funct7} &
\multicolumn{1}{c|}{rs2} &
\multicolumn{1}{c|}{rs1} &
\multicolumn{1}{c|}{funct3} &
\multicolumn{1}{c|}{rd} &
\multicolumn{1}{c|}{opcode} \\
\hline
7 & 5 & 5 & 3 & 5 & 7 \\
??????? & src2 & src1 & SHFL   & dest & OP    \\
% ??????? & src2 & src1 & SHFLW  & dest & OP-32 \\
\end{tabular}
\end{center}

\vspace{-0.4in}
\begin{center}
\begin{tabular}{R@{}R@{}R@{}S@{}R@{}O}
\\
\instbitrange{31}{27} &
\instbitrange{26}{20} &
\instbitrange{19}{15} &
\instbitrange{14}{12} &
\instbitrange{11}{7} &
\instbitrange{6}{0} \\
\hline
\multicolumn{1}{|c|}{imm[11:7]} &
\multicolumn{1}{c|}{imm[6:0]} &
\multicolumn{1}{c|}{rs1} &
\multicolumn{1}{c|}{funct3} &
\multicolumn{1}{c|}{rd} &
\multicolumn{1}{c|}{opcode} \\
\hline
5 & 7 & 5 & 3 & 5 & 7 \\
????? & mode & src & SHFLI  & dest & OP-IMM \\
\end{tabular}
\end{center}

% \vspace{-0.4in}
% \begin{center}
% \begin{tabular}{R@{}R@{}R@{}S@{}R@{}O}
% \\
% \instbitrange{31}{25} &
% \instbitrange{24}{20} &
% \instbitrange{19}{15} &
% \instbitrange{14}{12} &
% \instbitrange{11}{7} &
% \instbitrange{6}{0} \\
% \hline
% \multicolumn{1}{|c|}{imm[11:5]} &
% \multicolumn{1}{c|}{imm[4:0]} &
% \multicolumn{1}{c|}{rs1} &
% \multicolumn{1}{c|}{funct3} &
% \multicolumn{1}{c|}{rd} &
% \multicolumn{1}{c|}{opcode} \\
% \hline
% 7 & 5 & 5 & 3 & 5 & 7 \\
% ??????? & mode & src & SHFLIW  & dest & OP-IMM-32 \\
% \end{tabular}
% \end{center}


%%%%%%%%%%%%%%%%%%%%%%%%%%%%%%%%%%%%%%%%%%%%%%%%%%%%%%%%%%%%%%%%%%%%%%%%%%%%%%%%%%%%%%%%%%%

\section{Bit Extract/Deposit (\texttt{bext,\ bdep})}

This instructions implement the generic bit extract and bit deposit functions.
This operation is also referred to as bit gather/scatter, bit pack/unpack,
parallel extract/deposit, compress/expand, or right\_compress/right\_expand.

\texttt{bext} collects LSB justified bits to rd from rs1 using extract mask in rs2.

\texttt{bdep} writes LSB justified bits from rs1 to rd using deposit mask in rs2.

\input{bextcref-bext}

Implementations may choose to use smaller multi-cycle implementations of
\texttt{bext} and \texttt{bdep}, or even emulate the instructions in software.

Even though multi-cycle \texttt{bext} and \texttt{bdep} often are not fast
enough to outperform algorithms that use sequences of shifts and bit masks,
dedicated instructions for those operations can still be of great advantage in
cases where the mask argument is not constant.

For example, the following code efficiently calculates the index of the tenth
set bit in {\tt a0} using \texttt{bdep}:

\begin{verbatim}
  li a1, 0x00000200
  bdep a0, a1, a0
  ctz a0, a0
\end{verbatim}

For cases with a constant mask an optimizing compiler would decide when to use
\texttt{bext} or \texttt{bdep} based on the optimization profile for the
concrete processor it is optimizing for. This is similar to the decision
whether to use MUL or DIV with a constant, or to perform the same operation
using a longer sequence of much simpler operations.

\vspace{-0.3in}
\begin{center}
\begin{tabular}{S@{}R@{}R@{}S@{}R@{}O}
\\
\instbitrange{31}{25} &
\instbitrange{24}{20} &
\instbitrange{19}{15} &
\instbitrange{14}{12} &
\instbitrange{11}{7} &
\instbitrange{6}{0} \\
\hline
\multicolumn{1}{|c|}{funct7} &
\multicolumn{1}{c|}{rs2} &
\multicolumn{1}{c|}{rs1} &
\multicolumn{1}{c|}{funct3} &
\multicolumn{1}{c|}{rd} &
\multicolumn{1}{c|}{opcode} \\
\hline
7 & 5 & 5 & 3 & 5 & 7 \\
??????? & src2 & src1 & BEXT   & dest & OP       \\
??????? & src2 & src1 & BDEP   & dest & OP       \\
??????? & src2 & src1 & BEXTW  & dest & OP-32    \\
??????? & src2 & src1 & BDEPW  & dest & OP-32    \\
\end{tabular}
\end{center}


% \subsection{Justification}
%
% http://svn.clifford.at/handicraft/2017/permsyn/
%
% \subsection{References}
%
% http://programming.sirrida.de/bit\_perm.html\#gather\_scatter
%
% Hackers Delight, Chapter 7.1, ``Compress, Generalized Extract'' in
%
% https://books.google.com/books?id=iBNKMspIlqEC\&lpg=PP1\&pg=RA1-SL20-PA2\#v=onepage\&q\&f=false
%
% http://hackersdelight.org/
%
% https://github.com/cliffordwolf/bextdep
%
% http://palms.ee.princeton.edu/system/files/Hilewitz_JSPS_08.pdf

%%%%%%%%%%%%%%%%%%%%%%%%%%%%%%%%%%%%%%%%%%%%%%%%%%%%%%%%%%%%%%%%%%%%%%%%%%%%%%%%%%%%%%%%%%%

\section{Compressed instructions (\texttt{c.not,\ c.neg,\ c.brev})}

The RISC-V ISA has no dedicated instructions for bitwise inverse (\texttt{not})
and arithmetic inverse (\texttt{neg}). Instead \texttt{not} is implemented as
\texttt{xori\ rd,\ rs,\ -1} and \texttt{neg} is implemented as
\texttt{sub\ rd,\ x0,\ rs}.

In bitmanipulation code \texttt{not} and \texttt{neg} are very common operations. But
there are no compressed encodings for those operations because there is no \texttt{c.xori}
instruction and \texttt{c.sub} can not operate on \texttt{x0}.

Many bit manipulation operations that have dedicated opcodes in other ISAs
must be constructed from smaller atoms in RISC-V Bitmanip code. But
implementations might choose to implement them in a single micro-op using
macro-op-fusion. For this it can be helpful when the fused sequences are short.
\texttt{not} and \texttt{neg} are good candidates for macro-op-fusion, so
it can be helpful to have compressed opcodes for them.

Likewise \texttt{brev} (an alias for \texttt{grevi\ rd,\ rs,\ -1}, i.e. bitwise
reversal) is also a very common atom for building bit manipulation operations. So it
is helpful to have a compressed opcode for this instruction as well.

The compressed instructions \texttt{c.not,\ c.neg,\ c.brev} must be supported by
all implementations that support the C extension and Bitmanip.

\begin{table}[h]
\begin{small}
\begin{center}
\begin{tabular}{p{0in}p{0.05in}p{0.05in}p{0.05in}p{0.05in}p{0.05in}p{0.05in}p{0.05in}p{0.05in}p{0.05in}p{0.05in}p{0.05in}p{0.05in}p{0.05in}p{0.05in}p{0.05in}p{0.05in}l}
& & & & & & & & & & \\
                      &
\instbit{15} &
\instbit{14} &
\instbit{13} &
\multicolumn{1}{c}{\instbit{12}} &
\instbit{11} &
\instbit{10} &
\instbit{9} &
\instbit{8} &
\instbit{7} &
\instbit{6} &
\multicolumn{1}{c}{\instbit{5}} &
\instbit{4} &
\instbit{3} &
\instbit{2} &
\instbit{1} &
\instbit{0} \\
\cline{2-17}

&
\multicolumn{3}{|c|}{011} &
\multicolumn{1}{c|}{nzimm[9]} &
\multicolumn{5}{c|}{2} &
\multicolumn{5}{c|}{nzimm[4$\vert$6$\vert$8:7$\vert$5]} &
\multicolumn{2}{c|}{01} & C.ADDI16SP {\em \tiny (\sout{RES, nzimm=0})} \\
\cline{2-17}

&
\multicolumn{3}{|c|}{011} &
\multicolumn{1}{c|}{nzimm[17]} &
\multicolumn{5}{c|}{rd$\neq$$\{0,2\}$} &
\multicolumn{5}{c|}{nzimm[16:12]} &
\multicolumn{2}{c|}{01} & C.LUI {\em \tiny (\sout{RES, nzimm=0}; HINT, rd=0)} \\
\cline{2-17}

&
\multicolumn{3}{|c|}{011} &
\multicolumn{1}{c|}{0} &
\multicolumn{2}{c|}{00} &
\multicolumn{3}{c|}{rs2$'$/rd$'$} &
\multicolumn{5}{c|}{0} &
\multicolumn{2}{c|}{01} & C.NEG \\
\cline{2-17}

&
\multicolumn{3}{|c|}{011} &
\multicolumn{1}{c|}{0} &
\multicolumn{2}{c|}{01} &
\multicolumn{3}{c|}{rs1$'$/rd$'$} &
\multicolumn{5}{c|}{0} &
\multicolumn{2}{c|}{01} & C.NOT \\
\cline{2-17}

&
\multicolumn{3}{|c|}{011} &
\multicolumn{1}{c|}{0} &
\multicolumn{2}{c|}{10} &
\multicolumn{3}{c|}{rs1$'$/rd$'$} &
\multicolumn{5}{c|}{0} &
\multicolumn{2}{c|}{01} & C.BREV \\
\cline{2-17}

&
\multicolumn{3}{|c|}{011} &
\multicolumn{1}{c|}{0} &
\multicolumn{2}{c|}{11} &
\multicolumn{3}{c|}{---} &
\multicolumn{5}{c|}{0} &
\multicolumn{2}{c|}{01} & {\it Reserved} \\
\cline{2-17}
  
\end{tabular}
\end{center}
\end{small}
\end{table}


These three instructions fit nicely in the reserved space in C.LUI/C.ADDI16SP.
They only occupy $0.1\%$ of the $\approx15.6$ bits wide RVC encoding space.

%%%%%%%%%%%%%%%%%%%%%%%%%%%%%%%%%%%%%%%%%%%%%%%%%%%%%%%%%%%%%%%%%%%%%%%%%%%%%%%%%%%%%%%%%%%

\section{Bit-field extract pseudo instruction ({\tt bfext})}
\label{bfext}

Extract the continuous bit field starting at {\tt pos} with length {\tt len}
from {\tt rs} (with $\texttt{pos}>0$, $\texttt{len}>0$, and
$\texttt{pos}+\texttt{len}\le\textrm{XLEN}$).

\begin{verbatim}
  bfext rd, rs, pos, len   ->   slli rd, rs, (XLEN-pos-len)
                                srli rd, rd, (XLEN-len)
\end{verbatim}

If possible, an implementation should fuse this two shift operations into a single
macro-op. (Some implementers have raised concerns about the lack of a dedicated
bit field extract instruction with large immediate, especially for implementations
that can not fuse instructions into macro-ops and/or implementations that do
not support compressed instructions. See Chapter~\ref{bfxp} for a brief discussion
of why no such instruction is part of the RISC-V Bitmanip Extension, and a
short separate extension proposal that adds such an instruction.)
