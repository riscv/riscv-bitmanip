\chapter{RISC-V XTernarybits Extension}
\label{tbits}

This chapter describes four additional bit manipulation instructions {\tt cmix}, {\tt cmov},
{\tt fsl}, and {\tt fsr} instructions that require three source operands. Those instruction
are not included in XBitmanip because only instructions with one or two source operands are
included in the base XBitmanip spec for increased compatibility with existing RISC-V integer
pipelines.

All four instructions can be encoded in the same single minor opcode, for example in the
brownfield within the floating point fused-multiply-add instructions.

\section{Conditional mix ({\tt cmix})}

The {\tt cmix rd, rs1, rs2, rs3} instruction selects bits from {\tt rs1} and {\tt rs2} based
on the bits in the control word {\tt rs3}. It is equivalent to the following sequence.

\begin{verbatim}
  and rd, rs1, rs3
  andc t0, r2, rs3
  or rd, rd, t0
\end{verbatim}

Using {\tt cmix} a single butterfly or stage can be implemented in only two
instructions. Thus, arbitrary bit-permutations can be implemented using only
18 instruction (32 bit) or 22 instructions (64 bits).

\section{Conditional move ({\tt cmov})}

The {\tt cmov rd, rs1, rs2, rs3} instruction selects {\tt rs1} or {\tt rs2} based
on the LSB bit in the control word {\tt rs3}.

\section{Funnel shift ({\tt fsl}, {\tt fsr})}

The {\tt fsl rd, rs1, rs2, rs3} instruction creates a $2\cdot\textrm{XLEN}$ word
by concatenating rs1 and rs2 (with rs1 in the MSB half), rotate-left-shifts that
word by the amount indicated in the $log_2(\textrm{XLEN})+1$ LSB bits in rs3, and
then writes the MSB half of the result to rd.

The {\tt fsr rd, rs1, rs2, rs3} instruction creates a $2\cdot\textrm{XLEN}$ word
by concatenating rs1 and rs2 (with rs1 in the LSB half), rotate-right-shifts that
word by the amount indicated in the $log_2(\textrm{XLEN})+1$ LSB bits in rs3, and
then writes the LSB half of the result to rd.

